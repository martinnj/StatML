\section{II.1.2}
The training error on the transformed data is $14\%$ and the test error on
the transformed data is $21.05\%$. This is exactly the same as it is on the
un-transformed data, as the standardization of the data makes no difference
for Linear Discriminant Analysis. 

LDA discovers a number of functions which compute a posterior probability of the input, 
which in rough terms is equivalent to calculating the probability that a given input comes 
from that particular distribution (which is Gaussian in our case). In the end all that matters is
which of the functions produce the largest posterior probability, which is a
relative score compared to the other functions. Standardizing the data has no effect on this decision,
as the separating bounds between the probabilities remain the same relative to one another, standardized 
data or no.
